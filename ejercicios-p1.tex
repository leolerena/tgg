\documentclass[11pt]{article}
\usepackage[utf8]{inputenc}
\usepackage[spanish]{babel}
\usepackage{fancyhdr}
\usepackage{amsmath,amsfonts,amsthm,xcolor,amssymb,mathtools}
\pagenumbering{gobble}
\usepackage{geometry}\geometry{margin=1in}
\usepackage{hyperref}
\usepackage{amssymb}
\usepackage{eufrak}
\usepackage{tikz-cd}
\usepackage{tcolorbox}
\usepackage{enumitem}% http://ctan.org/pkg/enumitem
\setlist[itemize]{noitemsep, topsep=0pt}
\usepackage{verbatim} %Comentar una sección a la vez, más cómodo que usar %
\usepackage{sectsty}



\colorlet{chulo}{blue!70!purple}
\colorlet{rojo}{red!65!black}




\subsectionfont{\color{chulo!60!black} }
\sectionfont{\color{chulo!30!black} }


%Tipografía fancy
%\usepackage{mathpazo}
\usepackage{tgtermes}

%%%%%%%%%%%%%  Teoremas  %%%%%%%%%%%%%%%%%

\theoremstyle{plain}
\newtheorem{teo}{Teorema}
\newtheorem{propo}{Proposición}
\newtheorem{lema}[teo]{Lema}
\newtheorem*{ej}{Ejercicio}


%Definiciones
\theoremstyle{definition}
\newtheorem*{definition}{Definition}
\newtheorem*{sol}{Solución}
\newtheorem*{lemm}{Lema}

% Observaciones
\theoremstyle{remark}
\newtheorem{obs}{Observación}

%Para poner una afirmación.
\newenvironment{claim}[1]{\par\noindent\underline{Afirmación:}\space#1}{}
\newcommand\coker{\text{coker} \hspace{0.1cm} \phi}
\newcommand\im{\text{im} \hspace{0.1cm} \phi}
\newcommand\NN{\mathbb{N}}
\newcommand\VV{\mathbb{V}}
\newcommand\WW{\mathbb{W}}
\newcommand\ZZ{\mathbb{Z}}
\newcommand\RR{\mathbb{R}}
\newcommand\CC{\mathbb{C}}
\newcommand\B{\mathscr{B}}
\newcommand\A{\mathscr{A}}
\newcommand\G{\mathscr{G}}
\newcommand\X{\mathfrak{X}}
\newcommand\g{\mathfrak{g}}
\DeclarePairedDelimiter\abs{\lvert}{\rvert}
\DeclarePairedDelimiter\norm{\lVert}{\rVert}
\DeclarePairedDelimiter\ip{\langle}{\rangle}
\DeclareMathOperator{\id}{id}
\DeclareMathOperator{\GL}{GL}
\DeclareMathOperator{\SL}{SL}
\DeclareMathOperator{\Diff}{Diff}
\let\O\relax
\DeclareMathOperator{\O}{O}
\DeclareMathOperator{\U}{U}
\newcommand\eps{\varepsilon}
\newcommand\ol{\overline}
\newcommand\fder[2][]{\frac{\partial^{#1}}{\partial #2}}
\newcommand\rder[2]{\frac{\partial{#1}}{\partial #2}}
\newcommand\dd{\mathrm{d}}
\newcommand{\red}{\textcolor{red}}
\newcommand\smatrix[1]{\left(\begin{smallmatrix}#1\end{smallmatrix}\right)}

%opening
\usepackage{fancyhdr}
\pagestyle{fancy}
\lhead{Primer entrega de teoría geométrica de grupos.} % Left Header
\rhead{\thepage} % Right Header

\usepackage{subfiles} % Best loaded last in the preamble

\title{\color{red!55!black} Primer entrega de teoría geométrica de grupos.}
\author{Leopoldo Lerena.}
\date{2do Cuatrimestre de 2021.}

\begin{document}
\maketitle
Los siguientes lemas serán de utilidad en la solución del ejercicio.
\begin{lema}\label{fgsiipropio}
	Sea $G$ un grupo. Sea $S$ su conjunto de generadores. La métrica $d_S$ en $G$ es \textit{propia} si y solo si $S$ es finito.
\end{lema}
\begin{proof}
	Si suponemos que la métrica es propia, esto quiere decir que todas las bolas de radio finito tienen finitos elementos, entonces en particular la bola centrada en $1$ de radio $1$,
	\[
	B_1(1) = \{g | d(g,1)=1\}
	\]
	que está formada por los generadores $s \in S$ resulta ser finita. Esto nos dice que es finitamente generado.
	
	\medskip
	
	En cambio si suponemos que es finitamente generado, cualquier bola $B_r(g)$ como conjunto tiene una sobreyección,
	\[
	F: s_1 \dots s_r \in (S \cup S^{-1})^{r} \mapsto gs_1 \dots s_r \in B_r(g)
	\]
	donde $(S \cup S^{-1})^{r}$ es un conjunto finito dado que $S$ lo es. Esto prueba que toda bola de radio finito es finita, por lo que la métrica es propia.
\end{proof}
\bigskip
\begin{lema}\label{coro-svarc-milnor}[Práctica 1 - ejercicio 17]
	Sea $G$ un grupo finitamente generado, $H$ un subgrupo de índice finito, luego resulta que $H$ es finitamente generado y es quasi-isométrico a $G$.
\end{lema}
\begin{proof}
	Este resultado es un corolario del teorema de Svarc-Milnor. 
	
	Dado que $G$ al ser un grupo finitamente presentado resulta ser un espacio métrico de longitud, debemos ver que la acción de $H$ en $G$ por traslación a izquierda es geométrica. Se trata de una acción cocompacta dado que el cociente es finito por hipótesis y por lo tanto es compacto. A su vez la acción de $H$ es por isometrías dado que si $d(g,g')=n$, esto equivale a que
	\[
	g = g' s_1 \dots s_n
	\]
	para $\{s_i\}_i$ generadores de $G$, luego 
	\[
	hg = hg' s_1 \dots s_n.
	\]
	Esto nos dice que $d(hg,hg')=n$ para cualquier $h \in H$. Para finalizar de chequear las hipotesis debemos ver que la acción es propia. Para eso sea una bola $B_r(g)$ que es finita dado que el grupo es finitamente generado usando el resultado \ref{fgsiipropio}. En este caso como la acción es por isometrías resulta que
	\[
	hB_r(g) = B_r(hg).
	\]
	En particular observemos que si $hg \notin B_{2r}(g)$ debe ser que
	\[
	B_r(hg) \cap B_r(g) = \emptyset.
	\]
	De esta manera obtuvimos que
	\[
	|\{ h \in H: hB_r(g) \cap B_r(g) = \emptyset \}| \le |B_{2r}(g)| < \infty,
	\]
	concluyendo que la acción de $H$ en $G$ es propia. Con esto terminamos de ver que la acción es geométrica por lo que usando Svarć-Milnor resulta que $H$ es finitamente generado y que la inclusión
	\[
	H \xhookrightarrow{\iota} G
	\]
	es una quasi-isometría como queríamos ver.
\end{proof}
\bigskip
\begin{tcolorbox}[colback=teal!25!white,colframe=teal!75!black]
	\begin{ej}[Práctica 1 - ejercicio 16]
		Sea $\phi : \Gamma_1 \to \Gamma_2$ morfismo entre grupos finitamente generados. Probar que 
		\begin{itemize}
			\item [a)] Si $\phi$ es un embedding quasi-isométrico entonces $\ker \phi$ es finito.
			\item [b)] $\phi$ es quasi-isometría si y solo si $\ker \phi$ y $\coker$ son finitos.
		\end{itemize}
	\end{ej}
\end{tcolorbox}

\begin{sol}
	Fijemos un conjuntos de generadores $S$ de $\Gamma_1$.
	
	\begin{itemize}
		\item [a)] Si suponemos que $\phi$ es un embedding quasi-isométrico, es decir que su imagen no necesariamente es quasi densa, veamos que $\ker \phi$ es finito. Por definición de embedding quasi isométrico tenemos que existen constantes $\epsilon, \lambda \in \RR$ tales que para $g \in \ker \phi$,
		\[
		\lambda^{-1} d(\phi(g), \phi(1)) - \epsilon \le d(g,1)  \le \lambda d(\phi(g), \phi(1)) + \epsilon
		\]
		dado que $g \in \ker \phi$ resulta que $\phi(g)=1$, por lo que esta desigualdad nos dice que
		\[
		d(g,1) \le \epsilon.
		\]
		Dado que el grupo $\Gamma_1$ está finitamente generado esto nos dice por el resultado \ref{fgsiipropio} que solo existen finitos $g \in \ker \phi$. Con esto concluimos que $\ker \phi$ es finito.
		
		\item [b)] Probemos primero que si $\phi$ es una quasi-isometría resulta que $\ker \phi$ y $\coker$ son finitos. Por lo visto en el ítem anterior nos quedaría ver que en este caso $\coker$ es finito. Esto quiere decir que existen finitas coclases
		\[
		|\coker| =  |\{h \ker \phi, \hspace{0.2cm} h \in \Gamma_2\}| < \infty.
		\]
		Dado que la imagen de $\phi$ es quasi densa en $\Gamma_2$ debe ser que existe una constante $R$ tal que para todo $h \in \Gamma_2$
		\[
		d(\im, h ) < R.
		\]
		Reutilizando el lema \ref{fgsiipropio} obtenemos que la cantidad de coclases debe ser finita. Esto nos dice que $\coker$ es finito.
		
		\medskip
		
		Veamos ahora la recíproca. En este caso supongamos que $\ker \phi$ y $\coker$ son finitos y debemos verificar que $\phi$ es una quasi-isometría. Lo primero que vamos a ver es que la corestricción de $\phi$ a su imagen es una quasi-isometría. Esto se debe a que la imagen de $\phi$ es quasi-isométrica con $\Gamma_2$ por el lema \ref{coro-svarc-milnor} ya que tiene índice finito y $\Gamma_2$ es un grupo finitamente presentado. De una manera más formal lo que vimos es que la función
		\[
		\iota_{\im}: \im \to \Gamma_2
		\]
		es una quasi-isometría.
		
		\medskip
		
		El problema se reduce entonces a ver que 
		\[
		\phi^{\im}: \Gamma_1 \to \im
		\]
		es una quasi-isometría. Dado que la composición de quasi-isometrías es una quasi-isometría veríamos que
		\[
		\phi = \iota_{\im} \circ \phi^{\im} 
		\]
		es una quasi-isometría. De este lema obtenemos que el subgrupo $\im$ es finitamente generado, por lo tanto consideremos un conjunto de generadores $T$.
		
		\medskip
		
		Para ver que la función $\phi$ corestringida es una quasi-isometría debemos ver que es un embedding quasi-isométrico y que su imagen es quasi-densa. Al ser sobreyectiva solo debemos encargarnos de ver que es un embedding quasi-isométrico. Para eso debemos encontrar constantes $\lambda, \epsilon$ tales que para $g, g' \in \Gamma_1$ valga que
		\begin{equation}\label{ec-qi}
		\lambda^{-1} d(\phi(g), \phi(g')) - \epsilon \le d(g,g') \le \lambda d(\phi(g), \phi(g') + \epsilon.
		\end{equation}
		Dado un elemento $t$ del conjunto de generadores $T$, consideremos algún elemento en su preimagen que denotaremos $g_t$.
		Afirmamos que las siguientes elecciones de constantes cumplen lo pedido. Por un lado
		\[
		\epsilon = \max \{ d(1,g) : g \in \ker \phi \}
		\]
		y por otro lado tomemos
		\[
		\lambda = \max \{  d(1,g_t): t \in T \}
		\]
		donde las distancias son las respectivas a los finitos generadores de los grupos. Veamos que se satisface la ecuación \ref{ec-qi}. Por un lado sabemos que si
		\[
		d(\phi(g),\phi(g')) = n
		\]
		luego por definición de la distancia
		\[
		\phi(g) = \phi(g') t_1 \dots {t_n}
		\]
		para $t_i \in T$. Dado que $\phi$ es un morfismo de grupos resulta que
		\[
		\ol g \coloneqq g^{-1}g' (g_{t_1} \dots g_{t_n})^{-1} \in \ker \phi
		\]
		por lo que reacomodando nos queda la siguiente escritura,
		\[
		g' = g \ol g (g_{t_1} \dots g_{t_n})^{-1}.
		\]
		De manera que cada elemento $g_{t_i}$ tiene su distancia acotada por $\lambda$ y como hay exactamente $n$ de estos, resulta que usando las cotas elegidas,
		\[
		d(g',g) \le \lambda n + \epsilon 
		\]
		como queríamos ver.
	\end{itemize}
	
	
	\qed
\end{sol}

\line(1,0){500}

\bigskip


\begin{lema}\label{grafo-metrico}
	Dado un grafo $X$, si para todos los vértices $v$ existe una constante $\delta_v > 0$ tal que
	\[
	\lambda(e) > \delta_v
	\]
	para toda arista $e$ incidente en $v$ luego resulta que $X$ es métrico. 
\end{lema}

\begin{proof}
	Para ver que es un espacio métrico debemos chequear que $d$ se trata de una métrica. Para eso veamos que
	\[
	d(x,y)=0 \iff x=y.
	\]
	Sean dos puntos $x,y$ tales que $d(x,y)=0$. Veamos dependiendo de qué tipos de puntos son que $x=y.$
	
	Si ambos puntos no son vértices del grafo a la vez resulta que están sobre la misma arista caso contrario la distancia no podría ser 0. En este caso como las aristas son isométricas con intervalos de $\RR$ debe ser que $x=y.$
	
	En el caso que ambos sean vértices si utilizamos la hipótesis del enunciado debe ser que
	\[
	d(x,y) > \delta_x > 0.
	\]
	Esto nos dice que no pueden tener distancia nula en tal caso.
\end{proof}

\begin{tcolorbox}[colback=teal!25!white,colframe=teal!75!black]
	\begin{ej}[Práctica 6 - ejercicio]
		Un grafo con shapes finitos $X$ es métrico, completo y geodésico.
	\end{ej}
\end{tcolorbox}

\begin{sol}
	
	Para ver que es métrico como $\{ \lambda(e_n) \}_n$ es finito por hipótesis, si usamos el lema \ref{grafo-metrico} obtenemos que el espacio $X$ es métrico.
	
	\medskip
	
	Es geodésico cuando existe una geodésica que une dos puntos cualesquiera en el grafo. Dado que es un espacio de longitud, la distancia entre dos puntos $x,y$ se mide como
	\[
	d(x,y) = \inf \{ l(c) : c(0)=x, c(1)=y \}
	\]
	donde $c$ es una curva formada por aristas del grafo, que une a ambos puntos. Como el grafo tiene shapes finitas, resulta que si considero las curvas sin backtracking (que tienen mayor longitud) me quedo con finitas longitudes de curvas que unen a ambos puntos. En particular el ínfimo resulta ser un mínimo \footnote{Esto no quiere decir que sea únicamente geodésico porque si bien tiene shapes finitos puede haber infinitas aristas.}. Esto nos dice que existe una curva $c$ que minimiza la longitud. Esta curva resulta ser una geodésica, por lo tanto el grafo resulta ser geodésico.
	
	\medskip
	
	Para finalizar debemos ver que se trata de un espacio completo. Sea una sucesión de Cauchy $(x_n)_n$ debemos ver que converge a $x \in X$. Al ser una sucesión de Cauchy podemos tomar $\delta$ tal que
	\[
	\delta < \min \{ e \in \mathcal E | \lambda(e)/4 \}.
	\]
	El mínimo existe por que tiene shapes finitos. En tal caso existe $n_0$ tal que para todo $n,m > n_0$ vale que 
	\[
	d(x_n, x_m) \le \delta
	\] 
	por ser una sucesión de Cauchy. Por la distancia elegida debe haber un único vértice $w$ que diste menos de $\delta$ a la sucesión. En el caso que la sucesión tenga infinitos términos en una arista, al ser ésta compacta tiene una subsucesión convergente y por lo tanto $x_n$ es convergente. En el caso que no aparezca infinitas veces en una arista va a converger al vértice porque sino caso contrario existe $\epsilon$ tal que
	\[
	d(x_n, w) > \epsilon
	\]
	para todo $n \le n_0$. Como solo existe un vértice $w$ a distancia menor que $\delta$ y podemos tomar $\epsilon < \delta$, debe ser que $x_n$ cae dentro de una arista y por el argumento anterior tiene una subsucesión convergente y por lo tanto $x_n$ converge. 
	Si no existiese tal vértice tendría que la sucesión cae dentro de una arista que es completa por ser isométrica con un intervalo en $\RR$.
	
	
	\qed
\end{sol}




	
\end{document}