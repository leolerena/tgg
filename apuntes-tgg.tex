\documentclass[11pt]{article}
\usepackage[utf8]{inputenc}
\usepackage[spanish]{babel}
\usepackage{fancyhdr}
\usepackage{amsmath,amsfonts,amsthm,xcolor,amssymb,mathtools}
\pagenumbering{gobble}
\usepackage{geometry}\geometry{margin=1in}
\usepackage{hyperref}
\usepackage{amssymb}
\usepackage{eufrak}
\usepackage{tikz-cd}
\usepackage{tcolorbox}
\usepackage{enumitem}% http://ctan.org/pkg/enumitem
\setlist[itemize]{noitemsep, topsep=0pt}
\usepackage{verbatim} %Comentar una sección a la vez, más cómodo que usar %
\usepackage{sectsty}

\colorlet{chulo}{blue!70!purple}
\colorlet{rojo}{red!65!black}


\subsectionfont{\color{chulo!60!black} }
\sectionfont{\color{chulo!30!black} }


%Tipografía fancy
%\usepackage{mathpazo}
\usepackage{tgtermes}

%%%%%%%%%%%%%  Teoremas  %%%%%%%%%%%%%%%%%

\theoremstyle{plain}
\newtheorem{teo}{Teorema}
\newtheorem{propo}{Proposición}
\newtheorem{lema}[teo]{Lema}
\newtheorem*{ej}{Ejercicio}


%Definiciones
\theoremstyle{definition}
\newtheorem*{deff}{Definición}
\newtheorem*{sol}{Solución}
\newtheorem*{lemm}{Lema}

% Observaciones
\theoremstyle{remark}
\newtheorem{obs}{Observación}

%Para poner una afirmación.
\newenvironment{claim}[1]{\par\noindent\underline{Afirmación:}\space#1}{}
\newcommand\coker{\text{coker} \hspace{0.1cm} \phi}
\newcommand\im{\text{im} \hspace{0.1cm} \phi}
\newcommand\fg{finitamente generado }
\newcommand\NN{\mathbb{N}}
\newcommand\VV{\mathbb{V}}
\newcommand\WW{\mathbb{W}}
\newcommand\ZZ{\mathbb{Z}}
\newcommand\RR{\mathbb{R}}
\newcommand\CC{\mathbb{C}}
\newcommand\B{\mathscr{B}}
\newcommand\A{\mathscr{A}}
\newcommand\G{\mathscr{G}}
\newcommand\X{\mathfrak{X}}
\newcommand\g{\mathfrak{g}}
\DeclarePairedDelimiter\abs{\lvert}{\rvert}
\DeclarePairedDelimiter\norm{\lVert}{\rVert}
\DeclarePairedDelimiter\ip{\langle}{\rangle}
\DeclareMathOperator{\id}{id}
\DeclareMathOperator{\GL}{GL}
\DeclareMathOperator{\SL}{SL}
\DeclareMathOperator{\Diff}{Diff}
\let\O\relax
\DeclareMathOperator{\O}{O}
\DeclareMathOperator{\U}{U}
\newcommand\eps{\varepsilon}
\newcommand\ol{\overline}
\newcommand\fder[2][]{\frac{\partial^{#1}}{\partial #2}}
\newcommand\rder[2]{\frac{\partial{#1}}{\partial #2}}
\newcommand\dd{\mathrm{d}}
\newcommand{\red}{\textcolor{red}}
\newcommand\smatrix[1]{\left(\begin{smallmatrix}#1\end{smallmatrix}\right)}

%opening
\usepackage{fancyhdr}
\pagestyle{fancy}
\lhead{Teoría geométrica de grupos.} % Left Header
\rhead{\thepage} % Right Header

\usepackage{subfiles} % Best loaded last in the preamble

\title{\color{red!55!black} Teoría geométrica de grupos.}
\author{Leopoldo Lerena.}
\date{2do Cuatrimestre de 2021.}

\begin{document}
\maketitle
\section{Crecimiento de grupos.}
Sea $G$ un grupo nilpotente \fg nos gustaría entender cuál es su función de crecimiento. Un grupo nilpotente de \emph{clase} $k$ es tal que $C^{k+1} G = 1$. Para eso recordemos estas definiciones. 
\begin{deff}
	La \emph{dimensión homogenea} de un grupo nilpotente de rango $k$ es el siguiente número natural,
\begin{equation*}
	d = \sum_{i=0}^{k} i \ rg(C^i G/C^{i+1}G).
\end{equation*}
\end{deff}

En cierta manera este número es como una característica de Euler para un grupo nilpotente. El teorema importante es el siguiente,


\medskip
\begin{teo}[Bass-Guivar'ch]
	$G$ nilpotente y \fg con dimensión $d$ luego resulta que $\beta_G  \sim X^d$.
\end{teo}
\begin{proof}[Demo 1, versión más algebraica]
	La idea es hacer inducción en la clase del grupo nilpotente. Para eso notemos lo siguiente.
	\begin{itemize}
		\item \emph{Caso base.} Esto sale facilmente porque en tal caso es un grupo abeliano y como tal tiene que ser $\ZZ^d$ más la parte de torsión finita pero esto justamente nos dice que $\beta_G \sim x^d$. 
		\item \emph{Caso $k-1 \implies k$}. Veamos de construirnos algún grupo tal que tenga clase menor que $k$. Para eso lo natural es usar que todo los cocientes de grupos nilpotentes son nilpotentes con clase estrictamente menor. Consideremos el siguiente subgrupo $H=C^k G$ tal que el grupo $G/H$ es nilpotente de clase $k-1$ y tiene dimensión $d_1 = d - km_k$. 
		
		Notemos que si el grupo $H$ es finito en tal caso como $m_k = rg \  H$ resulta que nos da exactamente $m_k=0$. Equivalentemente esto nos dice que $d_1=d$. Por la hipótesis inductiva resulta que $\beta_{G/H} \sim x^d$ y por el resultado de la práctica que $G/H \underset{q.i}{\sim} G$ concluímos este caso.
		
		Veamos el caso que el grupo $H$ es infinito. Para eso debemos verificar que valen las dos cotas como funciones cuasidominadas una a la otra.
	\end{itemize}
\end{proof}



 
\end{document}